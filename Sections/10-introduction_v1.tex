\section{Introduction}
On-device processing is emerging as a vital component of modern human detection and tracking systems as an approach to ensure privacy. The ability to detect and track humans in real-time is crucial for a wide range of applications, from security surveillance to visitor analytics in cultural institutions. However, the deployment of such systems raises concerns about privacy and data security, particularly in sensitive environments like museums and aquariums. This thesis details the development of a privacy-preserving human localization system. The developed system was then deployed and tested in a aquarium in Denmark. 

A key issue with on-device processing where the images are deleted after inference is the inability to validate inference accuracy. With large benchmark datasets, one may evaluate model performance, but it may be hard to know if the performance metrics still apply when the models are deployed to a different, realistic environment. Most pre-trained models online are trained on datasets with general photographer-type images, while deployment scenearios are typically vastly different. 

\subsection{Background and Motivation}
The method of human detection and tracking in public spaces has evolved significantly over the past decade, driven by advancements in computer vision and machine learning. Traditional surveillance systems relied on centralized processing, where video feeds were transmitted to a remote server for manual human analysis. However, this approach raised privacy concerns as it involved transmitting raw video data over the network, potentially exposing sensitive information. Additionally, traditional systems required a human to manually analyze the video feed, which was time-consuming, prone to errors, lacking scalability, and not privacy-preservant. Recognizing these issues, this thesis proposes a shift towards \textit{on-device} processing to enhance privacy and efficiency. This approach is particularly pertinent for sensitive environments such as museums and aquariums, where privacy preservation is paramount.

To demonstrate the feasibility and effectiveness of on-device human detection and localization in a practical and realistic setting, two devices were deployed in the aquarium of "Fiskeri og Søfartsmuseet" in Esbjerg, Denmark. This setting was chosen to closely monitor and address the unique challenges presented by indoor, slightly dark-lit environments. A dataset of 3397 images 
% (todo this is 1st 2nd and 3rd iteration images) 
was collected (with consent of the subjects) and labeled. This dataset was then used to evaluate the performance of several object detector machine learning models. The data was also used to fine-tune the models. The best performing model was then deployed to the device to collect anonymous data on visitors for 1 month. The data was then visualized in multiple ways, including heatmaps and peak visitation hours. 

The project demonstrates how to overcome challenges of dark-lit environments without the use of expensive night vision cameras. A Raspberry PI 4 with a camera V2.1 module and a pre-trained YOLO object detection model version 9\footnote{YOLOv9 is not the state of the art (SOTA), but it was selected due to its usability and familiarity. The SOTA Co-DETR, which performs better on the COCO dataset, was considered but found impractical for this project due to its size and complexity in fine-tuning.} was used. The Raspberry Pi NoIR camera module was evaluated to determine its efficacy in enhancing human detection under low-light conditions\footnote{The "no" in NoIR signifies it's lack of an infrared filter, giving the ability to look in the dark \textbf{with infrared lightning}.}. Despite its potential, no significant improvement in performance was observed without additional infrared lighting. The obtained dataset from the aquarium is available and free to use at \href{https://drive.google.com/drive/folders/1_JXkpCqhaTc95XMjBc--Bkt0tH0Kdp-4?usp=sharing}{this Google Drive}.




\subsection{Scope}
The project had a wide scope, containing elements of edge-device deployment, machine learning and data science, requiring knowledge and development in all. The project encompassed numerous interdisciplinary steps that required significant time and effort, which are detailed in Section \ref{sec:methodology}. To reduce the workload, emphasis was placed on the development of a system that could be easily replicated and deployed in other environments. The model development and fine-tuning was expensive without an easily accessible GPU to train the models on. The models were thus fine-tuned on a cloud service, with challenges with regards to resource limits.  

The experiments in the aquarium were conducted on individuals that had not given written consent to being in images. It was therefore important that no sharp, privacy-intruding images were stored or uploaded. An already-developed system for secure communication with the Raspberry PIs were borrowed from \textit{HallMonitor}, a company in Esbjerg already having solutions for on-device processing. The borrowed system set up a secure shell (SSH) tunnel, also known as port forwarding, to communicate with devices. Such a secure system of communication is necessary to ensure privacy when using devices able to capture images of unknowing individuals. Due to the system being borrowed and HallMonitors intellectual property, the implementation of secure communication with edge devices is not included in this thesis. Additionally, no alternative approaches were explored, to keep the scope of the project manageable and to maintain a minimal level of security equal to that of HallMonitor.

Due to the wide scope of this project, less-than-favorable time was spent exploring better ways to fine-tune models. A larger number of machine learning models, including different variations of backbones and hyperparameters, could have been deployed to measure the differences in performance. This thesis includes the exploration of YOLOv3, YOLOv9, and DETR. Co-DETR, the current best-performing model on the COCO dataset, is also mentioned. It was not fully implemented as it proved time-consuming to set up and rendered unneccessary for the scope of this project. 

The scope of this project and its contribution to the field falls between showing a holistic implementation of how such a system may be developed and deployed, and the effects of certain choices during the development on the resulting outcome. For example, fine-tuning a YOLO model by freezing the backbone and fine-tuning the head for 5, 15 and 50 epochs showed an increasing decline in performance of the model with more training epochs. 

The contribution of this project to the field lies in demonstrating a comprehensive implementation of a privacy-preserving human localization system. This includes detailing the developmental choices and their impacts on the system's performance, highlighting the trade-offs between different technological approaches. For instance, fine-tuning a YOLO model with varying epochs on just the head while freezing the backbone revealed a counterintuitive decline in performance with increased training duration. This may be due to multiple reasons, further explored in the results section (\ref{sec:results_fine_tuning}). In addition, the project explores dataset quality importance for correct evaluation of object detectors to real-world applications. 

\subsubsection{Research Questions}
\label{sec:research_questions}
\begin{enumerate}
	\item What are the main privacy risks associated with traditional human localization systems in public spaces?
	\item How effectively can on-device processing mitigate these privacy concerns compared to centralized processing methods?
	\item What are the trade-offs in detection performance when implementing on-device processing?
	\item How does the validity of object detection model evaluations change when using data specifically from the intended deployment environment compared to using generic datasets?
\end{enumerate}

\subsubsection{Research Objectives}

\paragraph{Primary}
\begin{enumerate}
	\item Develop a privacy-preserving human localization system using on-device processing to minimize data transmission and enhance data privacy.
	\item Evaluate the performance of the developed system in a real-world setting (e.g., an aquarium) to determine its effectiveness and reliability.
	\item Investigate the effects of the dataset quality in fine-tuning of models on the performance.
	\item Assess the impact of deployment-specific data on the accuracy and validity of object detection model evaluations, by comparing performance metrics with those obtained using generic datasets.
\end{enumerate}

\paragraph{Secondary}
\begin{enumerate}
	\item Compare the privacy and performance impacts of on-device processing against traditional centralized methods. 
	\item Investigate the feasibility of deploying the developed system in other public spaces to enhance visitor analytics and security.
	\item Develop a visualization tool to analyze and interpret the collected data for practical applications.
\end{enumerate}
% todo sammenligne dataen de har på hvor mange besøkende på en dag. Har de data om når på dagen det er travlest? Vet du hvor mange som er i innom akvariet?

\subsection{Structure}
\label{sec:structure}
This thesis is organized into several chapters that systematically explore the development, deployment, and evaluation of a privacy-preserving human localization system. Each chapter is structured to progressively build upon the foundational concepts introduced in this section:

\begin{enumerate}
    \item \textbf{Chapter 1: Introduction} - Provides an overview of the project, including the background, motivation, and scope of the research.
    \item \textbf{Chapter 2: Literature Review} - Reviews existing literature on human localization systems, privacy concerns in surveillance technologies, and on-device processing techniques.
    \item \textbf{Chapter 3: Methodology} - Details the methodologies employed in developing the system, including hardware configurations, software architectures, and data collection protocols.
    \item \textbf{Chapter 4: System Implementation and Deployment} - Describes the practical implementation of the system within the aquarium and discusses the deployment challenges and solutions.
    \item \textbf{Chapter 5: Results and Discussion} - Presents the findings from the deployment, analyses the data collected, and discusses the implications of the results.
    \item \textbf{Chapter 6: Conclusion and Future Work} - Summarizes the research findings, discusses the contributions of the thesis, outlines the limitations, and suggests directions for future research.
\end{enumerate}

Each chapter is designed to provide detailed insights into the respective aspects of the project, ensuring a comprehensive understanding of the system's development, challenges, and performance within a real-world setting.
