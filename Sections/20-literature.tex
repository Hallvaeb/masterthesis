\section{Literature Review}
The advent of "modern" object detection has enabled more sophisticated and automated methods for understanding visitor engagement and flow in cultural institutions. This literature review aims to explore the current state of research on object detection and visitor behaviour analysis in cultural institutions, focusing on privacy-preserving techniques, dataset specialization for enhanced object detection accuracy, and case studies of technology implementation. 

\subsection{Visitor Behavior Analysis in Cultural Institutions}
Studies on traditional methods for analyzing visitor behavior (surveys, manual counting, direct observation) and their use cases and limitations.


\citeauthor{la2017museumbehaviouranalysis} conducted a study on the behaviour of museum visitors, and the perceived value of their findings to museum curators. They found that the use of technology for visitor behaviour analysis was generally well-received by museum curators, and that the data collected could be used to improve the visitor experience. They also avoided the use of manual counting and surveys, by giving the visitors wearable RFID trackers, communicating their position to the system when close to one of several beacons deployed at positions deemed important by the museum curators. The study was able to provide insights into the visitors' behaviour, such as the most popular exhibits, the average time spent in the museum, and the most common paths taken by visitors. The biggest draw back of such a system is having to give the visitors wearable trackers, which can be unfavourable. TDOO: finne ut om de spurte deltakere om hva de synes om å bha med seg devices.

On the usability of a visitor behavior analysis systems, \cite{la2017museumbehaviouranalysis} found split opinions. Administrators and department heads were generally more enthustistic, while the museum curators were generally more sceptical: "A temporary exhibition won’t
change after you deploy it, and understanding how it is used by the public would not help me in my next exhibition, since they are each very different. My main reason to analyze behavior would be to satisfy my curiosity." On the contrary, one museum administrator stated: "Some exhibitions cost a fortune. We really need to know if this expenditure was worthwhile" displaying how opinions can vary greatly depending on the role of the individual in the museum.


\subsection{Privacy-Preserving Technologies in Surveillance}
Exploration of techniques for ensuring privacy in surveillance, such as blurring, pixelating faces etc.. 

!TODO insert chapter from forstudien om hva folk tolker som privacy preservant og ikke. Poengtere at det mest personvernvennlige er å ikke lagre data i det hele tatt.!

\subsubsection{Privacy and Ethics}
Research addressing the ethical implications of surveillance in public spaces, including visitor perceptions and legal considerations.



\subsection{Introduction to Object Detection and Visitor Tracking}
Review of the evolution of object detection and tracking technologies, including the transition from traditional methods to modern computer vision techniques. What are some of their challenges, and how have they been addressed? Specifically focusing on suboptimal lighting conditions.

*Har fått yolov9 oppe og kjøre på pcen så tenker å kortfattet oppsummere hvordan yolo-algoritmen er ulik noen andre algoritmer, hva yolo er brukt til, og hvordan v9 skiller seg fra tidligere versjoner. Ta for seg hva vekter er i denne konteksten og at vektene bestemmes av treningsdata, som gir en glidende overgang til \ref{sec:dataset_specialization}.

!evne til å forbedre mørke scener i mobiltelefonteknologi vil kanskje også over noen år bli overført til mer lavnivåsystemer.

Primary objective demonstrate feasability and effectiveness of on-device human detection and tracking in a practical and realistic setting.
Burde være en sekundary goal her kanskje å lage en heatmap.
Avvening om lysstyrke, bevegelse, bilder, sett fra menneske og maskin.


\subsubsection{Yolov9 Object Detection}
The YOLO (You Only Look Once) object detection algorithm is a popular choice for real-time object detection due to its speed and accuracy. YOLO processes images in a single pass, making it faster than traditional object detection algorithms that require multiple passes. YOLO divides the image into a grid and predicts bounding boxes and class probabilities for each grid cell. The YOLOv9 is an improved version of the original YOLO algorithm, incorporating various enhancements to improve detection accuracy and speed. The YOLOv9 model used for this project uses weights that has been pre-trained on the COCO dataset. This is a large dataset of ... images, with 80 different classes (i.e. objects). To improve the model, removing all other classes than persons would likely allow for a smaller model which could mean a smaller size weights file for the resulting model. Training a model from scratch, however, is a time consuming process which requires a lot of data and processing.




\subsubsection{Dark-Lit Environments}
Being able to detect and locate people in dark-lit environments have been previously attempted, usually for security concerns in public spaces. \citeauthor{pa2020PersonDetectionNightTimeFLIR} developed a system for detecting people in dark-lit environments using a convolutional neural network. They modified the three input channels which usually take RGB to take as input instead (i) the original infrared image, (ii) a difference image from the previous frame, and (iii) a background subtraction mask. Their dataset is vastly different from the setting for this thesis, as the individuals in their photos were far away from the cameras. However, they found that their system was able to detect people in dark-lit environments with an accuracy of 90\%. This is a promising result, as it shows that it is possible to detect people in dark-lit environments using infrared imaging and CNNs. They used FLIR cameras, which make images from heat. Doing inferences on pure infrared images may be harder because the infrared radiation may be less prevalent than the heat a FLIR camera may capture. The FLIR cameras are expensive, and thus not considered viable for this project. 
\cite{pa2020PersonDetectionNightTimeFLIR}. 




\subsubsection{Effectiveness of Training Dataset Specialization}
\label{sec:dataset_specialization}
Comparison of various algorithms performance enhancement when data has been optimized for the use case. What is the role of a training dataset in the task of determining the weights in a yolov9 artificial neural network, and how can would special training data optimize the weights for a specific use case?



\subsection{Summary of Literature Review}
A summarization of the current state of research, and where my thesis aims to contribute.
