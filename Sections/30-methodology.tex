\section{Methodology}
Two cameras were deployed in a room of acquariums at "Fiskeri- og Søfartsmuseet" in Esbjerg to take images for building a specialized dataset and to evaluate the effects of developing a highly specialized detector rather than using a general.

Camera setup:
Hardware camera can be tweaked by screwing the lens with a mechanical tool to modify it's aperture, which influences it's depth focus. 
Aperture mechanical setting (camera focus adjustment), depth control. Default not found... images well sharp enough... 50cm to infinity...

I was getting the following errorS:
mma l:
mal_uc_port_enable: failed to enable port
uc.null_sink: in:0(OPQV): ENOSPC
mal: mal_port_enable: failed to enable connected
port (uc.null_sink: in:0COPQU) )0x256daf0 CENOSPC)
mmal connection enable:
Error: Failed
output port couldn't be enabled
to enable connection: Out of resources
Which was solved by allocating more memory to the GPU by going through raspi-config Performance Options > GPU Memory

The awb_gains control seems to not set before after capturing an image. Thus, setting these values and then checking, rather than setting them, capturing and then checking leads to confusing results.

Dataset obtainment
Dataset was build by capturing images while no visitors were present in the acquarium. Due to the constraint to only operate within opening hours when the facility was open to everyone, a way to cancel image capturing was needed in the case if someone entered the room. One of the goals of the dataset was to have the images taken from the same angle as the device will be used in the future. The device was therefore mounted in the corner of the room, and ssh was used to access the device remotely from a pc in the acquarium. Then, a script was ran to capture an image per second and storing it on the SD card in the device. The choice to store the image locally rather than transmitting it was to not have to worry about data transmission costs and issues. The first iteration of image capture was made with non-optimized camera configurations, turning to the brightness setting of the picamera python module to get the images to a sufficient brightness. Still, some images fell short to the auto focus of the camera focusing on the bright acquariums and thus rendering the rest of the image rather dark. However, these images serve as a way of inspecting the impact of captured image quality on inference performance. 

Due to many technical difficulties the first few times images were being captured for the dataset, only the developer and author of this thesis is present in the images\footnote{Initially, an attempt was made to pass MQTT messages as a way to initialize image capture so multiple cameras could be deployed in several locations, thus speeding up and simplifying the image capturing process. This was discarded due to technical difficulties related to efficiently stopping the image capturing. For a different solution, a recommendation would be to implement a way of communicating to multiple devices however, so one could obtain all images from deployment locations in one shooting. For this single-deployment angle and area project, however, the approach with ssh-ing into the device worked fine.}. The acquarium could not provide free tickets for additional participants in the experiments, and the cost of bringing friends to enrichen the dataset would then have to be optimized in terms of time spent at the site.

The detector needs to know the ground truth when training and validating on the obtained data. This can be obtained by manually labeling the data. However, a more scientific, robust and scaleable way of labeling is to have the detector do the heavy lifting. Therefore, the dataset was first inferenced by the detector. The data was then manually validated, and finally the images for which the detector did not find any persons was manually labeled. The time to understand the tools and develop this pipeline was similar to what it would take to manually label all the images, but can now be used for future applications also. The approach to label the images is described in section: \ref{sec:labeling}.

After the ground truth was identified for all the dataset images, it was then used to evaluate the general-purpose yolov3 and yolov9. Results are discussed in \ref{sec:results}. The data was then used to train the detectors. The training process is described in section \ref{sec:training}. The trained models were then deployed to the device to evaluate the inference speed and accuracy. The deployment process is described in section \ref{sec:deployment}.

To visualize the improvements and highlight the areas in the image benefitting most from detector improvements, heat maps were generated. The process of generating heat maps is described in section \ref{sec:heatmaps}.

\subsection{Labeling}
\label{sec:labeling}
Ground truth values for the dataset must be obtained before improving the model and evaluating it's effect. 

"Label Studio" was used to label the images. First, the dataset was predicted with a yolo v9 model trained on the COCO dataset. Predictions were mostly decent, but some needed small tweakings and in some cases the persons were not discovered. The detector had close to zero hallucinations due to a sufficiently high confidence rate of 0.5 (: for ), but in some cases the fish were identified as human.

\subsection{Ethical Considerations}
In the deployment of advanced machine learning technologies for visitor localization and engagement analysis, this research proactively addresses privacy concerns through the implementation of image obscuration techniques. These methods ensure that no personally identifiable information is captured or communicated, thus significantly reducing privacy risks associated with visitor tracking in cultural spaces such as museums and aquariums.

\subsubsection{Privacy by Design}
At the forefront of our ethical approach is the principle of "privacy by design." This concept involves integrating privacy into the development and operation of our tracking technologies from the outset, rather than as an afterthought. By employing image obscuration techniques, such as real-time pixelation or silhouette generation, we ensure that the visual data processed by our system remains anonymous. This method effectively eliminates the possibility of identifying individual visitors from the captured data, thereby safeguarding their privacy.

The application of these privacy-preserving techniques negates the need for explicit consent from visitors for two primary reasons. First, the anonymization process occurs instantaneously as the data is captured, meaning no identifiable information is ever stored or analyzed. Second, the focus of the research is on aggregate behavior patterns rather than individual actions, further distancing the study from privacy concerns.

\subsubsection{Ethical Use and Data Protection}
Ensuring the ethical use of technology extends beyond privacy considerations to include the responsible handling and protection of any data generated by the system. Although the data is anonymized, we are committed to maintaining high standards of data protection. This includes secure data storage, limiting access to authorized personnel, and employing robust data management policies that comply with relevant data protection laws and guidelines.

The utilization of anonymization techniques also reflects our commitment to minimizing any potential impact on visitor behavior and the overall museum or aquarium experience. By ensuring that the tracking system is unobtrusive and does not compromise privacy, we aim to maintain the integrity of the visitor experience, allowing individuals to engage with exhibits without concern for their privacy.

\subsubsection{Transparency and Accountability}
While the technical approach effectively addresses privacy concerns, maintaining transparency about the use and purpose of tracking technologies is still essential. Information about the tracking system and its privacy-preserving nature will be made available to visitors, ensuring they are informed about how data is used to enhance the visitor experience.

Furthermore, the project will adhere to an ongoing ethical review process, ensuring that all aspects of the research remain aligned with ethical best practices and respond to evolving technological and societal standards.

In summary, by prioritizing privacy through the use of image obscuration techniques and adopting a comprehensive ethical framework, this research aims to advance the understanding of visitor engagement in a manner that is both innovative and respectful of individual privacy rights. This approach sets a precedent for the ethical application of machine learning technologies in cultural institutions, balancing the benefits of visitor behavior analysis with the imperative of protecting privacy.



\subsection*{Notes}
Tried to download/use model from Roboflow, but either image has to be sent to an API which would not retain privacy, or the device has to host an API itself to run the inference... Seems unlikely to be the most preferable solution, as the device would have to set up the service and run it locally. Possibly an interesting solution would be to do this with multiple devices. This supports the master-slave pattern of having multiple weaker computers and have them send to the stronger unit. Setting up private TCP connection between the weaker units and the strong unit and have the images sent to the stronger, so it can detect on them and send information etc... How many weak units do we need in order to make it profitable to have a strong GPU unit to do the processing? This whole systems sounds to be complicating processes, not making the product modular and easy-to-use. Includes a lot of connection/networking to make the weaker units find and connect to strong, physically close device.
    This task would mean setting up a strong device to host a network to which the weak units might connect to, and send images to. The issue is whenever images are sent, a lot of transmission is used... But the model takes image input size of 416x416. Would it be similar to just downscale the image before sending, or would this give the model less detail to work with?

Will now run several models on datasets from the web, i.e. the CrowdHuman dataset to see their accuracies. Will then deploy the models to device in acquarium to see if the best-performing model is an option in terms of size and inference speed. If it is preferable, I will attempt to increase it's accuracy by accumulating and annotating a specialized dataset for that setting, and training the final layers on the data. Can this be done with a 