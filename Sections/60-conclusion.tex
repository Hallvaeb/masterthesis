\section{Conclusions}
\label{sec:conclusions}

\paragraph{Summary of Findings}
This thesis has explored the viability and ethical implications of developing on-device person detection systems, with a strong emphasis on privacy preservation. The research demonstrates that on-device processing significantly enhances data security and privacy, effectively mitigating the risks associated with person localization systems in public spaces, while still enabling the analysis of visitor behavior.

\paragraph{Model Evaluation and Dataset Relevance}
Through extensive experimentation, it has been established that the validity of object detection model evaluations improves when tested on data specific to their intended deployment environment. This was evident from the significant variation in scores when the models were tested on the specific deployment dataset vs the generic datasets they've been evaluated on before. This finding underscores the importance of using relevant and high-quality datasets for model evaluation to achieve accurate and practical results. The analysis of this thesis lack evaluations of fine-tuned models on generic datasets for a complete assessment of the fine-tuned models accuracies on deployment-specific test data relative to generic data.

\paragraph{Technological Innovation and Ethical Framework}
The thesis delved into the ethical considerations surrounding the deployment of person localization systems. It highlighted the potential risks of mass public control and emphasized the need for a balanced approach that weighs the benefits of technological advancements against the risks to privacy and individual freedoms. The consumer trend to prioritize convenience over privacy in smart home applications hightlight the need for regulations such as GDPR and NIS2 to control technology. The application of utilitarian principles, along with insights from modern philosophical discourse, serves as a framework for assessing the ethical deployment of these technologies. This thesis underscores the importance of a holistic approach that balances technical efficacy with privacy concerns.

\paragraph{Practical Implementation}
The practical implementation at the "Fiskeri og Søfartsmuseet" aquarium showcased the technical feasibility and utility of on-device detection technologies in real-world settings. The successful collection and analysis of visitor data demonstrated the system's potential for enhancing crowd management and operational efficiency in public venues. Importantly, the project's focus on anonymized data collection and privacy-preserving measures exemplifies how similar person localization systems can be integrated into societal infrastructures without infringing on individual privacy.

\paragraph{Future Directions and Policy Recommendations}
The rapid pace of AI development necessitates that regulators possess a nuanced understanding of the technologies they seek to govern, ensuring that laws and policies are both protective, up-to-date, and conducive to innovation. Developers should be encouraged to pursue creative solutions while adhering to ethical guidelines. A balance between innovation and ethics must be achieved through collaborative efforts between technologists, ethicists, policymakers, and the public. This thesis contributes to the ongoing discourse by bridging the knowledge gap, and can help foster collaborative and communicative efforts in this area.

\paragraph{Final Remarks}
\textit{Overall, this thesis contributes valuable insights into the development and implementation of privacy-preserving person localization systems. By addressing both technical and social dimensions, it provides a comprehensive understanding of the challenges and opportunities for these systems. The insights advocate for continued innovation, informed by rigorous ethical standards and regulations, to ensure that technological advancements serve societal well-being while safeguarding individual privacy.}